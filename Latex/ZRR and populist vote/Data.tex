\section{Data} \label{data-section}

The electoral data comes from the French Ministry of the Interior. Municipality characteristics are from the French censuses of 1990, 1999, and 2007. Table \ref{tab:descriptive} reports descriptive statistics of the control variables in 1990, all of which are known to influence electoral behavior. Additionally, I gathered geographical characteristics of municipalities, such as altitude and size of the area. I computed the distance to the closest agglomeration (defined as a locality with a population size in 1990 above the 9th decile of the overall population size distribution in 1990). Using satellite data from the National Geographic Institute (IGN), [YS: reference] I also collected information on the density of fences (haie) to measure bocage areas and the density of vines for wine-producing regions. Lastly, I obtained data on the density of Organizations of Public Interest (OPI) from the \textit{Journal officiel des associations et fondations d'entreprise} (JOAFE) as a measure of Social Capital (in the sense of \cite{putnam2000}).

% Descriptive Statistics
\begin{table}[!h]
\centering
\caption{\label{tab:descriptive}Descriptive Statistics in 1990}
\centering
\resizebox{\ifdim\width>\linewidth\linewidth\else\width\fi}{!}{
\begin{tabular}[t]{>{\raggedright\arraybackslash}p{7cm}ccc}
\toprule
\multicolumn{1}{c}{ } & \multicolumn{3}{c}{Groups} \\
\cmidrule(l{3pt}r{3pt}){2-4}
Variable & Never Treated & Treated after 1995 & Treated in 1995\\
\midrule
\addlinespace[0.3em]
\multicolumn{4}{l}{\textbf{Past elections}}\\
\hspace{1em}Vote share for FN in 1988 & \makecell{14.22 \\ (6.16)} & \makecell{11.95 \\ (5.55)} & \makecell{10.67 \\ (6.01)}\\
\hspace{1em}FN vote share in 1995 & \makecell{16.53 \\ (6.35)} & \makecell{14.02 \\ (6.20)} & \makecell{11.63 \\ (6.62)}\\
\addlinespace[0.3em]
\multicolumn{4}{l}{\textbf{Employment}}\\
\hspace{1em}Unemployed (\%) & \makecell{8.66 \\ (5.58)} & \makecell{9.02 \\ (7.17)} & \makecell{8.79 \\ (9.16)}\\
\hspace{1em}In the labor force (\%) & \makecell{25.17 \\ (30.82)} & \makecell{24.31 \\ (18.06)} & \makecell{25.25 \\ (17.75)}\\
\hspace{1em}Agriculture (\%) & \makecell{7.84 \\ (11.09)} & \makecell{17.87 \\ (17.57)} & \makecell{25.67 \\ (22.58)}\\
\hspace{1em}Independant (\%) & \makecell{8.23 \\ (6.12)} & \makecell{8.38 \\ (8.23)} & \makecell{9.04 \\ (10.52)}\\
\hspace{1em}Intermediate occupations (\%) & \makecell{19.36 \\ (8.62)} & \makecell{14.65 \\ (10.32)} & \makecell{13.55 \\ (12.14)}\\
\hspace{1em}Clerical (\%) & \makecell{22.85 \\ (8.49)} & \makecell{18.96 \\ (10.83)} & \makecell{18.00 \\ (13.15)}\\
\hspace{1em}Manual (\%) & \makecell{33.39 \\ (13.38)} & \makecell{35.30 \\ (15.81)} & \makecell{29.27 \\ (17.75)}\\
\addlinespace[0.3em]
\multicolumn{4}{l}{\textbf{Demographics}}\\
\hspace{1em}Population & \makecell{10550.59 \\ (83850.47)} & \makecell{799.29 \\ (2190.72)} & \makecell{446.88 \\ (1195.58)}\\
\hspace{1em}Foreigners (\%) & \makecell{2.89 \\ (3.81)} & \makecell{1.62 \\ (2.66)} & \makecell{1.87 \\ (2.92)}\\
\hspace{1em}Ages 20-40 (\%), men & \makecell{18.72 \\ (3.88)} & \makecell{17.86 \\ (5.18)} & \makecell{16.79 \\ (6.43)}\\
\hspace{1em}Ages 20-40 (\%), women & \makecell{17.80 \\ (3.70)} & \makecell{16.04 \\ (4.67)} & \makecell{14.50 \\ (5.64)}\\
\hspace{1em}Age ratio young/old (\%) & \makecell{81.10 \\ (22.60)} & \makecell{69.24 \\ (22.28)} & \makecell{58.33 \\ (24.22)}\\
\hspace{1em}Population density & \makecell{3.98 \\ (12.54)} & \makecell{0.56 \\ (1.51)} & \makecell{0.26 \\ (0.52)}\\
\hspace{1em}Population change in p.p. 1980-1990 & \makecell{0.20 \\ (0.34)} & \makecell{0.05 \\ (0.19)} & \makecell{-0.02 \\ (0.21)}\\
\hspace{1em}Vacant housing (\%) & \makecell{6.66 \\ (3.84)} & \makecell{8.87 \\ (4.78)} & \makecell{10.21 \\ (5.94)}\\
\hspace{1em}OPI per 1,000 inhabitants & \makecell{2.81 \\ (0.80)} & \makecell{3.02 \\ (0.73)} & \makecell{3.26 \\ (0.79)}\\
\hspace{1em}Taxable income per capita  (log) & \makecell{10.58 \\ (0.25)} & \makecell{10.43 \\ (0.23)} & \makecell{10.35 \\ (0.23)}\\
\bottomrule
\end{tabular}}
\end{table}
\clearpage
\begin{table}[!h]
\centering
\caption{Descriptive Statistics in 1990 (continued)}
\centering
\resizebox{\ifdim\width>\linewidth\linewidth\else\width\fi}{!}{
\begin{threeparttable}
\begin{tabular}[t]{>{\raggedright\arraybackslash}p{7cm}ccc}
\toprule
\multicolumn{1}{c}{ } & \multicolumn{3}{c}{Groups} \\
\cmidrule(l{3pt}r{3pt}){2-4}
Variable & Never Treated & Treated after 1995 & Treated in 1995\\
\midrule
\addlinespace[0.3em]
\multicolumn{4}{l}{\textbf{Education}}\\
\hspace{1em}No diploma (\%) & \makecell{21.23 \\ (8.60)} & \makecell{25.91 \\ (10.07)} & \makecell{26.44 \\ (11.74)}\\
\hspace{1em}Academic (\%) & \makecell{5.79 \\ (4.13)} & \makecell{3.97 \\ (3.67)} & \makecell{4.13 \\ (4.48)}\\
\hspace{1em}Highschool (\%) & \makecell{6.88 \\ (3.37)} & \makecell{5.83 \\ (4.00)} & \makecell{6.23 \\ (5.05)}\\
\hspace{1em}Technical (\%) & \makecell{15.91 \\ (4.90)} & \makecell{14.09 \\ (5.68)} & \makecell{13.66 \\ (7.01)}\\
\addlinespace[0.3em]
\multicolumn{4}{l}{\textbf{Geography}}\\
\hspace{1em}Altitude & \makecell{4.89 \\ (1.00)} & \makecell{5.03 \\ (0.85)} & \makecell{5.76 \\ (0.76)}\\
\hspace{1em}Distance to closest agglomeration in meters (log) & \makecell{9.04 \\ (2.49)} & \makecell{10.18 \\ (0.64)} & \makecell{10.44 \\ (0.44)}\\
\hspace{1em}Area in km2 (log) & \makecell{6.92 \\ (0.83)} & \makecell{7.03 \\ (0.76)} & \makecell{7.23 \\ (0.77)}\\
\hspace{1em}Observations & 18000 & 6662 & 10937\\
\bottomrule
\end{tabular}
\begin{tablenotes}
\item \textit{Note: } 
\item Cells report the mean (first line) and standard deviation (in parentheses, second line). Some variables are taken from nearby years: taxable income (1994) and young/old ratio (1995).
\item[a] In the socio-economic database assembled by Julia Cag\'e and Thomas Piketty (2023), the average income per municipality is defined as the total income reported on tax declarations (before any deductions or allowances) divided by the total number of inhabitants (including children). Source: \url{https://www.unehistoireduconflitpolitique.fr/glossaire.html}, the website associated with the book by Julia Cag\'e and Thomas Piketty (2023): \textit{Une histoire du conflit politique. \'{E}lections et in\'{e}galit\'{e}s sociales en France, 1789--2022}, Paris, Le Seuil.
\end{tablenotes}
\end{threeparttable}}
\end{table}



The first notable element is that the proportion of agricultural workers is significantly higher in the municipalities treated in 1995. Additionally, these municipalities have smaller populations, more OPIs per capita, fewer manual workers (ouvriers), they experienced the harshest negative employment shock between 1982 and 1990, saw their population decreased between 1982 and 1990, and are situated at higher altitudes.