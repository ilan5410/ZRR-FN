\section{Discussion}\label{results}
\subsection{Socioeconomic causes of populism}

In the introduction, I argued that the ZRR program could mitigate the electoral support for the FN because it would affect the socioeconomic variables that, traditionally in the literature, are known to have influence on the populist vote. I compare the evolution of these variables between 1990 and 1999 (the two censuses) using the sharp discontinuity design that I used earlier. %Appendix \ref{??} present the discontinuity graphs on each of the residuals of the variables of interest, when I control for all fixed effects (county, department, region) and the set of other controls. %
Table \ref{tab:tab:effect_1999} shows no significant effect of the ZRR program on the 1999 socioeconomic variables. The slight increase in population is most certainly due to the gap we observed earlier between the two groups of comparison. %Table X shows that the change in population size between 1990 and 1999 is not affected by the program.
The results are consistent with those in \cite{BEHAGHEL20151}, which concludes that the ZRR program did not affect employment and firm creation.\footnote{One of the reasons to the program's failure is that, at the same time, an urban Enterprise-Zone was created and proved to be a large success. It might be that those urban EZ were in competition with the rural ones, and won.} I conclude that there is no clear evidence that the ZRR program had any noticeable effect on socio-economic outcomes, and therefore that such effects were unlikely to have been the mechanism explaining the ZRR effect on voting. %This is consistent with the finding of \cite{MAGALHAES2025103224}, according to whom there is no evidence that cultural and economic factors played a relevant role in the relationship between rurality and the radical right vote in Portugal. Instead, the political neglect seems to have contributed to a negative perception that fed, in turn, the support for the radical right.


%% Results on socio economic outcomes
\begin{table}
\centering
\caption{\label{tab:tab:effect_1999}Effect of the ZRR program on the 1999 socioeconomic variables}
\centering
\begin{tabular}[t]{llll}
\toprule
Outcome & 20 km & 10 km & 5 km\\
\midrule
Ages 20-40 (\%), men & -0.003 (0.001)* & -0.002 (0.001) & 0 (0.002)\\
Ages 20-40 (\%), women & -0.002 (0.001)* & -0.001 (0.001) & -0.001 (0.002)\\
Age ratio young/old (\%) & -0.011 (0.004)** & -0.003 (0.005) & 0.001 (0.007)\\
Population density & 0.022 (0.003)*** & 0.012 (0.002)*** & 0.004 (0.003)\\
Unemployed (\%) & 0 (0.002) & -0.002 (0.002) & -0.001 (0.003)\\
\addlinespace
In the labor force (\%) & 0.005 (0.003) & 0.004 (0.004) & 0.014 (0.005)**\\
Foreigners (\%) & 0.002 (0)*** & 0.001 (0.001)* & -0.001 (0.001)\\
Agriculture (\%) & 0.011 (0.002)*** & 0.008 (0.003)* & 0.003 (0.005)\\
Independant (\%) & 0.005 (0.002)*** & 0.002 (0.002) & 0.001 (0.003)\\
Intermediate occupations (\%) & -0.007 (0.002)** & -0.003 (0.003) & 0.001 (0.005)\\
\addlinespace
Clerical (\%) & 0.001 (0.002) & -0.001 (0.003) & 0.002 (0.005)\\
No diploma (\%) & 0.001 (0.002) & 0.001 (0.002) & 0.001 (0.003)\\
Academic (\%) & -0.002 (0.001) & 0 (0.001) & -0.002 (0.002)\\
Highschool (\%) & 0 (0.001) & -0.001 (0.001) & -0.001 (0.002)\\
Technical (\%) & -0.003 (0.001)* & -0.003 (0.002) & 0.002 (0.003)\\
\addlinespace
OPI per 1,000 inhabitants & 0.013 (0.006)* & 0.011 (0.008) & 0.004 (0.013)\\
Manual (\%) & -0.011 (0.003)*** & -0.009 (0.003)* & -0.006 (0.005)\\
Taxable income per capita  (log) & -0.024 (0.014) & -0.001 (0.018) & -0.017 (0.029)\\
log\_pop\_1999 & -0.003 (0.003) & 0 (0.003) & -0.001 (0.005)\\
nobs\_1999 & 20482 & 14423 & 9113\\
\bottomrule
\end{tabular}
\end{table}




\subsection{Signal Effect}

I now turn to the hypothesis of a signaling effect from the government, suggesting that the growing support for populist parties is fueled by a sense of political neglect. The central government in Paris sent a powerful signal in February 1995, just three months before the presidential elections, when the National Assembly approved the ZRR program. In Appendix \ref{app:1995signal}, I present the results of my estimations of the ZRR treatment signal on the vote share for the FN in 1995. I find a very small negative yet not significant effect, indicating that the signal channel is almost null. 

\subsection{Support for populism or anger against the incumbent?}


Until now, I considered the share of votes for the FN. Here, I want to understand whether the observed negative effects of the ZRR program stem from changes in the numerator (the absolute number of FN votes), the denominator (total votes), or a combination of both.

Voters may be attracted to the Front National party due to its pro-welfare stance, or perhaps it is the appeal of its populist rhetoric, which taps into desires for a more locally-centered, culturally homogeneous community. That would affect the numerator. Another possible motivation is the desire to punish incumbent politicians, which would affect the denominator.


[TODO: The results changed with the "new" sample, need to adapt this paragraph] The results displayed in Table \ref{tab:absolute-vote} suggest that there is no significant effect of the program on the log of the absolute number of votes for the FN (the numerator), although the effect seems negative (except for the municipalities in a 7,5 km range around the program frontier). Earlier, we found that there was no precise effect of the program on the turnout (Table \ref{tab:rdd_results_outcomes_later}). We conclude that the small effect of the program stems from a diminished support for the FN, that does not arise from the mobilization of "new" voters.


% absolute number of votes, different bandwidths

% Table created by stargazer v.5.2.3 by Marek Hlavac, Social Policy Institute. E-mail: marek.hlavac at gmail.com
% Date and time: Mon, Feb 02, 2026 - 23:46:47
\begin{table}[!htbp] \centering 
  \caption{ZRR effect on absolute number of votes (log) for FN, different bandwidths} 
  \label{tab:absolute-vote} 
\begin{tabular}{@{\extracolsep{5pt}}lccc} 
\\[-1.8ex]\hline 
\hline \\[-1.8ex] 
 & \multicolumn{3}{c}{\textit{Dependent variable:}} \\ 
\cline{2-4} 
\\[-1.8ex] & \multicolumn{3}{c}{FN2002abs} \\ 
 & Bandwidth = 20 km & Bandwidth = 10 km & Bandwidth = 5 km \\ 
\\[-1.8ex] & (1) & (2) & (3)\\ 
\hline \\[-1.8ex] 
 Treatment ZRR & $-$0.043$^{***}$ & $-$0.033$^{*}$ & 0.002 \\ 
  & (0.011) & (0.014) & (0.022) \\ 
  & & & \\ 
 Distance to Frontier & 0.00000 & 0.00000 & 0.00001$^{*}$ \\ 
  & (0.00000) & (0.00000) & (0.00000) \\ 
  & & & \\ 
 Constant & $-$4.785$^{***}$ & $-$4.502$^{***}$ & $-$4.335$^{***}$ \\ 
  & (0.226) & (0.260) & (0.356) \\ 
  & & & \\ 
\hline \\[-1.8ex] 
Controls & Yes & Yes & Yes \\ 
Department fixed effects & Yes & Yes & Yes \\ 
Region fixed effects & No & No & No \\ 
Observations & 19,210 & 13,519 & 8,537 \\ 
R$^{2}$ & 0.870 & 0.861 & 0.829 \\ 
Adjusted R$^{2}$ & 0.870 & 0.860 & 0.827 \\ 
\hline 
\hline \\[-1.8ex] 
\textit{Note:}  & \multicolumn{3}{r}{$^{*}$p$<$0.05; $^{**}$p$<$0.01; $^{***}$p$<$0.001} \\ 
 & \multicolumn{3}{r}{Standard errors are clustered at the canton level} \\ 
\end{tabular} 
\end{table} 


